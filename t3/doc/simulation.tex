\section{Simulation Analysis}
\label{sec:simulation}

\subsection{Envelope detector}
The simulated results of the envelope detector output are compared to the theoretical results as shown below in Figures \ref{fig:sim_env} and \ref{fig:theo_env}, respectively.


\begin{figure} [ht]
\centering
\begin{minipage}{.5\textwidth}
  \centering
  \includegraphics[width=0.9\linewidth]{../sim/transv2.pdf}
  \captionof{figure}{Simulated Envelope detector voltage output}
  \label{fig:sim_env}
\end{minipage}%
\begin{minipage}{.5\textwidth}
  \centering
  \includegraphics[width=0.9\linewidth]{venvlope.eps}
  \captionof{figure}{Theoritical Envelope detector voltage output}
  \label{fig:theo_env}
\end{minipage}
\end{figure}

The theoretical ripple in the envelope detector is considerably smaller than the simulated one, due to the approximations made in the theoretical diode model.




\pagebreak
\subsection{Output Voltage}
The simulated results of the voltage output are compared to the theoretical results as shown below in Figures \ref{fig:sim_vout} and \ref{fig:theo_vout}, respectively.

\begin{figure} [h]
\centering
\begin{minipage}{.5\textwidth}
  \centering
  \includegraphics[width=0.9\linewidth]{../sim/transv4.pdf}
  \captionof{figure}{Simulated voltage output.}
  \label{fig:sim_vout}
\end{minipage}%
\begin{minipage}{.5\textwidth}
  \centering
  \includegraphics[width=0.9\linewidth]{vout.eps}
  \captionof{figure}{Theoritical voltage output.}
  \label{fig:theo_vout}
\end{minipage}
\end{figure}

Once again, the theoretical ripple is considerably smaller than the simulated one, due to the approximations made in the theoretical diode model.

It is also notable that the output voltage values are always higher than both 12V and the simulation's values, which vary between higher and lower than 12V.  

\pagebreak
The voltage output errors ($V_{out}$ - 12) of the simulation and the theoretical analysis are shown above in Figures \ref{fig:sim_error} and \ref{fig:theo_error}, respectively.


\begin{figure}
\centering
\begin{minipage}{.5\textwidth}
  \centering
  \includegraphics[width=0.8\linewidth]{../sim/trans1.pdf}
  \captionof{figure}{Simulated voltage output error.}
  \label{fig:sim_error}
\end{minipage}%
\begin{minipage}{.5\textwidth}
  \centering
  \includegraphics[width=0.8\linewidth]{erro.eps}
  \captionof{figure}{Theoritical voltage output error.}
  \label{fig:theo_error}
\end{minipage}
\end{figure}

The output voltage error is always positive and higher than the simulation's, which varies between positive and negative.


Comparing the Voltage ripple and VDC of the simulated and theoretical analysis in tables \ref{tab:sim} and \ref{tab:tab1}, respectively:

\begin{table}[h]
  \centering
  \begin{tabular}{|l|r|}
    \hline    
    {\bf Name} & {\bf Value [V]} \\ \hline
    \input{../sim/passo2}
  \end{tabular}
  \caption{Simulated results. mean(v(4)) is the average outuput voltage and vecmax(v(4))-vecmin(v(4)) is the maximum value of ripple. The last value is the merit of the circuit.}
  
  \label{tab:sim}
\end{table}

\begin{table}[h]
  \centering
  \begin{tabular}{ |c|c|}
 \hline
 {\bf Name} & {\bf Value[V]} \\
 \hline\hline
  $V_{DC}$ & \partialinput{4}{4}{../mat/tab1.tex}\\
 \hline
 $V_{ripple}$ & \partialinput{9}{9}{../mat/tab1.tex} \\
 \hline
 \end{tabular}
  \caption{Theoretical values. $V_{DC}$ is the average outuput voltage.}
  \label{tab:tab1}
\end{table}

The relative error for the average output voltage is aproximately 0.713\%.
The relative error for the maximum ripple is aproximately -34.22\%. These errors are once again due to the approximations made in the theoretical diode model.

The merit value is 3.754540e-01.
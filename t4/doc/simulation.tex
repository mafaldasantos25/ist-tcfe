\section{Simulation Analysis}
\label{sec:simulation}

\subsection{Operating Point}
The simulated results of the operating point analysis of the amplifier circuit are shown below in Table \ref{tab:sim_op}.



\begin{table}[h]
  \centering
  \begin{tabular}{|l|r|}
    \hline    
    {\bf Name} & {\bf Value [V]} \\ \hline
    \input{../sim/iT4}
  \end{tabular}
  \caption{Simulated operating point results. }
  
  \label{tab:sim_op}
\end{table}


The relative error for the node voltages is aproximately -3.05\% for v(emit), -21.5\% for v(coll) and 20\% for V(emit2). These errors are considerably big, and are caused by the theoretical model used (BJT incrementl model).
We can see that the results for VCE-VBE for the NPN BJT and the results for VEC-VEB for the PNP transistor confirm that they are operating in the Forward Active Region, like is intended for this circuit.

\subsection{Frequency Analysis}
The simulated results for the output impedance, input impedance, lower and upper 3dB cut off frequencies and the voltage gain are shown in tables \ref{tab:sim_freq} and \ref{tab:sim_freq2}, for a frequency in the pass-band region.

Output Impedance:


\begin{table}[h]
  \centering
  \begin{tabular}{|l|r|}
    \hline    
    {\bf Name} & {\bf Value [Ohm]} \\ \hline
    \input{../sim/opimp}
  \end{tabular}
  \caption{Simulated output impedance result. }
  
  \label{tab:sim_freq}
\end{table}

Voltage Gain (dB), cut off frequencies, input impedance and merit:

\begin{table}[h]
  \centering
  \begin{tabular}{|l|r|}
    \hline    
    {\bf Name} & {\bf Value [db]/[Ohm]} \\ \hline
    \input{../sim/pT4}
  \end{tabular}
  \caption{Voltage Gain (dB), cut off frequencies, input impedance and merit simulated reults.For the variables with two values, only the first one is of interest (frequency analysis).}
  
  \label{tab:sim_freq2}
\end{table}

The relative errors of each one are:
Error(Voltage Gain)=26.4\%
Error(Input Impedance)=14.5\%
Error(Output Impedance)=60\%

These values are extremely high due to approximations made with the incremental BJT model.

\subsection{Gain's frequency response}

The plot of the Voltage Gain frequency response is shown below.

\begin{figure}[h] \centering
\includegraphics[width=0.5\linewidth]{../sim/trans1.pdf}
\caption{Voltage Gain frequency response.}
\label{fig:VG}
\end{figure}

In the theoretical analysis, only the pass-band gain value was obtained. We can see that even this value isn't constant in the pass-band. Before the lower cutt of frequency, we could compare the simulated result to a linear response of 20dB/dec. After the upper cutt of frequency, we can compare the graph to a linear response of -20dB/dec.


























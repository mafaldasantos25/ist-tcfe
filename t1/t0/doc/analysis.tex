\section{Theoretical Analysis}
\label{sec:analysis}

In this section, the circuit shown in Figure~\ref{fig:circuito_t1} is analysed
theoretically, using the mesh and nodal methods.

\subsection{Mesh Method}

The circuit consists of 4 meshes, with 7 resistors, 2 independent sources ($V_a$ for voltage source and $I_d$ for current source) and 2 dependent sources ($V_c$ for current controlled voltage source and $I_b$ for voltage controlled current source).

\begin{figure}[h] \centering
\includegraphics[width=0.5\linewidth]{circuito_T1_Malhas.pdf}
\caption{Mesh method analysis.}
\label{fig:malha}
\end{figure}

Apllying Kirchoff's Voltage Law (KVL) to the two meshes on the left ($I_a$ is the current for the top one, $I_c$ for the bottom one, $I_b$ for the top right one, $I_d$ for the bottom right, as ilustrated in Figure~\ref{fig:malha}, we get two equations:

\begin{equation}
  (R_4+R_3+R_1)I_a-R_3I_b-R_4I_c=0.
  \label{eq:kvl1}
\end{equation}

\begin{equation}
  -R_4I_a+(R_6+R_7-K_c+R_4)I_c=0.
  \label{eq:kvl2}
\end{equation}


The other two meshes have current sources, so we cannot apply KVL to them. Instead, we recognize the relation between the sources and the meshes' currents: $I_d$ is equal to the current coming from the $I_d$ source and the following equation: 

\begin{equation}
  I_b=K_bV_b.
\end{equation}

Now, because

\begin{equation}
  V_b=R_3(I_b-I_a).
\end{equation}
 
 we get:
 
 \begin{equation}
  I_b=K_bR_3(I_b-I_a).
  \label{eq:kvl3}
\end{equation}
 

The solution is obtained by solving Equations~(\ref{eq:kvl1}), ~(\ref{eq:kvl2}) and ~(\ref{eq:kvl3}) and is illustrated in Table~\ref{tab:mesh}.

\begin{table}[H]
  \centering
  \begin{tabular}{ |c|c|}
 \hline
 {\bf Name} & {\bf Value[A or V]} \\
 \hline\hline
  $@I_a$ & \partialinput{4}{4}{malhas.tex}\\
 \hline
  $@I_b$ & \partialinput{9}{9}{malhas.tex} \\
 \hline
 $@I_c$ & \partialinput{14}{14}{malhas.tex} \\
 \hline
 $I_d$ & \partialinput{19}{19}{malhas.tex} \\
 \hline
 $V_b$ & \partialinput{24}{24}{malhas.tex} \\
 \hline
 $V_c$ & \partialinput{29}{29}{malhas.tex} \\
 \hline
\end{tabular}
  \caption{Mesh Method. A variable preceded by @ is of type {\em current}
    and expressed in Ampere; other variables are of type {\it voltage} and expressed in
    Volt.}
  \label{tab:mesh}
\end{table}


\subsection{Nodal Method}

The circuit consists of 8 nodes. These were numbered from 0 to 7 as illustrated in the image below. The node 0 was chosen as the reference node. 

\begin{figure}[h] \centering
\includegraphics[width=0.5\linewidth]{circuito_T1_Nos.pdf}
\caption{Node method analysis.}
\label{fig:nos}
\end{figure}


Apllying Kirchoff's Current Law (KCL) to nodes 1, 2, 5 and 6 we get four equations:

\begin{equation}
  -V_1G_1-(V_2-V_1)G_2+(V_4-V_1)G_3=0.
  \label{eq:kcl1}
\end{equation}

\begin{equation}
  K_b(V_1-V_4)+(V_1-V_2)G_2=0.
  \label{eq:kcl2}
\end{equation}

\begin{equation}
  I_d+(V_4-V_5)G_5-K_b(V_1-V_4)=0.
  \label{eq:kcl3}
\end{equation}

\begin{equation}
  (V_3-V_6)G_6+(V_7-V_6)G_7=0.
  \label{eq:kcl4}
\end{equation}

In addition, because voltage source $V_a$ is connected between the reference node and a non reference node, we simply set the voltage at the non-reference node equal to the voltage of the voltage source.

\begin{equation}
  V_3=-V_a.
  \label{eq:kcl5}
\end{equation}

As the voltage source $V_c$ is between two non reference nodes then it forms a supernode whose analysis is done as following:

\begin{equation}
  (V_3-V_4)G_4+(V_1-V_4)G_3+(V_5-V_4)G_5-I_d+(V_6-V_7)G_7=0.
  \label{eq:kcl6}
\end{equation}

\begin{equation}
  (V_4-V_7)=K_c(V_3-V_6)G_6.
  \label{eq:kcl7}
\end{equation}


The solution is obtained by solving Equations~(\ref{eq:kcl1}), ~(\ref{eq:kcl2}), ~(\ref{eq:kcl3}), ~(\ref{eq:kcl4}), ~(\ref{eq:kcl5}), ~(\ref{eq:kcl6}) and ~(\ref{eq:kcl7}) and is illustrated in Table~\ref{tab:nodal}.

\begin{table}[h]
  \centering
  \begin{tabular}{ |c|c|}
 \hline
 {\bf Name} & {\bf Value[A or V]} \\
 \hline\hline
  $V_0$ & \partialinput{4}{4}{nos.tex}\\
 \hline
 $V_1$ & \partialinput{9}{9}{nos.tex} \\
 \hline
 $V_2$ & \partialinput{14}{14}{nos.tex} \\
 \hline
 $V_3$ & \partialinput{19}{19}{nos.tex} \\
 \hline
 $V_4$ & \partialinput{24}{24}{nos.tex} \\
 \hline
 $V_5$ & \partialinput{29}{29}{nos.tex} \\
\hline
 $V_6$ & \partialinput{34}{34}{nos.tex} \\
 \hline
 $V_7$ & \partialinput{39}{39}{nos.tex} \\
 \hline
 $V_b$ & \partialinput{44}{44}{nos.tex} \\
 \hline
 $V_c$ & \partialinput{49}{49}{nos.tex} \\
 \hline
 $@I_b$ & \partialinput{54}{54}{nos.tex} \\
 \hline
\end{tabular}
  \caption{Nodal Method. A variable preceded by @ is of type {\em current}
    and expressed in Ampere; other variables are of type {\it voltage} and expressed in Volt.}
  \label{tab:nodal}
\end{table}

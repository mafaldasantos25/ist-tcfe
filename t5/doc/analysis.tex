\section{Theoretical Analysis}
\label{sec:analysis}

In this section, the BandPass filter is analysed theoretically.

Since the filter is active, it's composed of a high-pass filter ($R_1$
and $C_1$), a low-pass filter ($R_2$ and $C_eq$) and a voltage
amplifier.

$C_eq$ is the approximate equivalent capacitance of $C_2$, $C_21$, $C_22$:

\begin{equation}
  C_eq=\frac{1}{\frac{1}{C_2}+\frac{1}{C_21}+\frac{1}{C_22}}.
  \label{eq:n1}
\end{equation}

The voltage gain will be the product of the gain in each part of the circuit:

\begin{equation}
  Gain=20*log10(abs(\frac{R_1}{\frac{1}{s*C_1}+R_1}*(1+\frac{R_4}{R_3})*\frac{1}{1+R_2*s*C_2})).
  \label{eq:n2}
\end{equation}

The central frequency is calculated using the lower cutt of frequency
and the upper one, which are the poles of the low and high pass
filters:
\begin{equation}
  central frequency=\sqrt{f_h*f_L}.
  \label{eq:n3}
\end{equation}

Finnaly the input and output impedances are calculated considering an
ideal op-amp model:

\begin{equation}
  input_impedance=\frac{1}{s*C_1}+R_1.
  \label{eq:n4}
\end{equation}

\begin{equation}
  output_impedance=\frac{1}{\frac{1}{R_2}+s*C_2}.
  \label{eq:n5}
\end{equation}

\subsection{Central frequency analysis}

First we will analyse the gain at central frequency.  
The results, using the equations shown above, are shown in Table  \ref{tab:tab1}.

\begin{table}[h]
  \centering
  \begin{tabular}{ |c|c|}
 \hline
 {\bf Name} & {\bf Value [Ohm]/[dB]/[Hz]} \\
 \hline
 $@Central frequency$ & \partialinput{4}{4}{../mat/tab1.tex} \\
 \hline
 $*Gain$ & \partialinput{9}{9}{../mat/tab1.tex} \\
 \hline
 $Z_{input}$ & \partialinput {14}{14}{../mat/tab1.tex}\\
 \hline
  $Z_{output}$ & \partialinput {19}{19}{../mat/tab1.tex}\\
 \hline
 
 \end{tabular}
  \caption{Theoretical values at central frequency. The gain is in dB. Values with @ are frequecies in Hz.} 
  \label{tab:tab1}
\end{table}


\subsection{Frequency Response}

The voltage gain's frequency response can be seen in figure \ref{fig:fr} and the plot of the Phase frequency response is shown in Figure \ref{fig:fphase}.

\begin{figure}[H]
\centering
  \includegraphics[width=0.5\linewidth]{fr.eps}
  \captionof{figure}{Gains's Frequency Response.}
  \label{fig:fr}
\end{figure}

\begin{figure}[H]
\centering
  \includegraphics[width=0.5\linewidth]{fphase.eps}
  \captionof{figure}{Phase Frequency Response.}
  \label{fig:fphase}
\end{figure}
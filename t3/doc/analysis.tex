\section{Theoretical Analysis}
\label{sec:analysis}

In this section, the output voltage, voltage ripple and envelope detector output of the circuit are all analysed theoretically.

\subsection{Envelope detector}

Using an ideal diode model, we plotted the time response of the output voltage, as shown in figure .

First we calculated $t_off$
\begin{equation}
  tOFF = (1/w) * atan(1/w/R/C).
  \label{eq:menor toff}
\end{equation}

The output voltage of the envelope detector varies accordingly with $t_off$:

For t<$t_off$
\begin{equation}
  vO(t) = v_s(t).
  \label{eq:menor toff}
\end{equation}

For t>$t_off$
\begin{equation}
  vO(t) = abs(A*cos(w*tOFF)*exp(-(t(i)-tOFF)/R/C)).
  \label{eq:maior toff}
\end{equation}

To simplify the calculations, we plotted for t>$t_off$ with the condition that if the full rectifier output was superior to $abs(A*cos(w*tOFF)*exp(-(t(i)-tOFF)/R/C))$, vO(t) would be equal to that output, so as not to calculate $t_ON$.

\subsection{Output Voltage}

We obtain the output voltage using a voltage regulator circuit.

First we calculate $v_e$ 

\begin{equation}
  v_e = vO-median(vO).
  \label{eq:ve}
\end{equation}

Then we use a voltage divider
vo = (rd/(rd+R)) * vs
\begin{equation}
  vo = (rd/(rd+R)) * vs
  \label{eq:vo}
\end{equation}

And lastly, we can calculate the voltage output

\begin{equation}
  Vout = VD + vo;
  \label{eq:Vout}
\end{equation}
